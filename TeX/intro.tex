\ath{Aðgerðagreining} eða aðgerðarannsóknir beinast að því að ákvarða hagkvæmustu leið til að framkvæma eitthvað innan fyrirtækja eða stofnana. Iðnaðarverkfræðingar fást oft slík rekstrar tengd verkefni, en aðgerðagreining takmarkast þó engan vegin við slík verkefni. 

Oft er um að ræða ákvarðanatöku þar sem flókin viðfangsefni eru sett fram sem bestunarverkefni. Lausn upphaflega verkefnisins felst þá í að finna hámark/lágmark á tilteknu falli, svokallað \ath{markfall}. 

Aðgerðagreining fellur undir hagnýta stærðfræði og samtvinnar m.a. tölfræði, líkindafræði, tölvunarfræði, ákvarðanafræði, biðraðafræði, leikjafræði, netafræði, hermun og bestun. Áherslan í þessari bók, verður mest lögð á líkanagerð og bestun. 

Aðgerðagreining eins og hún er stunduð í dag má rekja aftur til seinni heimstyrjaldarinnar þegar breskir og bandarískir vísindamenn voru fengnir til að finna hvernig ráðstafa mátti takmörkuðum auðlindum á hagkvæman máta. 

\section{Nokkur dæmi um bestunarverkefni} 
\begin{description} 
\item[Rekstur] Hámarka hagnað fyrirtækja (lágmarka skuldir?), hámarka afköst framleiðslulína, lágmarka kostnað við vörudreifingu (t.d. útkeyrslu og lagerhald). 
\item[Landbúnaður] Hámarka verðmæti uppskeru m.t.t. takmarkana á landrými, lágmarka kostnað við fóðurgjöf (t.d. kjúklinga eða svína). 
\item[Byggingarverkfræði] Lágmarka þyngd mannvirkja (t.d. háspennumöstur) sem uppfylla jafnframt hönnunarkröfur. 
\item[Umhverfisverkfræði] Koma mengun undir viðmiðunarmörk en lágmarka jafnframt kostnað við þær framkvæmdir. Hámarka hagnað af endur\-vinnslu. 
\item[Hagfræði] Hámarka þjóðarframleiðslu, taka þarf tillit til takmarkaðs vinnuafls, atvinnuleysis ofl. (Sjá t.d. \href{http://en.wikipedia.org/wiki/Wassily_Leontief}{Leontif}) 
\item[Fjarskiptaverkfræði] Bestun á úthlutun tíðnisviða í símkerfum. 
\item[Læknisfræði] Lágmarka skaða heilbrigðra líffæra í geislameðferð við krabba\-meini. 
\item[Fjármálaverkfræði] Val á hlutabréfum, lánastýring.
\end{description}

\begin{aths}Sjá fleiri tilvik um beitingu aðgerðagreiningar í töflu á bls. 4 í kennslubók. \end{aths}

\section{Ferli}

\begin{enumerate} 
\item Skilgreing verkefnis og gagnasöfnun. 
\item Stærðfræðilegt líkan útbúið sem fangar kjarna viðfangsefnisins. 
\item Tölvuforrit þróað til að vinna með líkanið. 
\item Prófun (sannreyning) líkans. Líkanið endurbætt ef nauðsyn krefur. 
\item Líkanið tekið í notkun -- yfirleitt í formi forrits. 
\end{enumerate} 
 
